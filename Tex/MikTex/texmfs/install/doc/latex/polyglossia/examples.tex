\documentclass[a4paper]{article}
\usepackage[no-math]{fontspec}
\usepackage{xltxtra,url}
\let\XeTeX\undefined
\let\XeLaTeX\undefined
\usepackage{polyglossia}
\usepackage{trace}
\setdefaultlanguage{french}
\setotherlanguage[variant=british,ordinalmonthday=false]{english}
\setotherlanguage[variant=poly]{greek}
\setotherlanguage[numerals=thai]{thai}
\setotherlanguage[locale=mashriq]{arabic}
\setotherlanguage[spelling=new,latesthyphen=true,babelshorthands=true]{german}
\setotherlanguages{latin,russian,turkish,polish,latvian,sanskrit,ukrainian,farsi,syriac,divehi,hebrew,amharic,nko}
\setotherlanguage[calendar=gregorian,numerals=western]{urdu}
\setmainfont{Linux Libertine O}
\defaultfontfeatures{Scale=MatchLowercase}
\setmonofont{Inconsolata}
\newfontfamily\arabicfont[Script=Arabic]{Amiri}
\newfontfamily\syriacfont[Script=Syriac]{Serto Jerusalem}
\newfontfamily\hebrewfont[Script=Hebrew]{Ezra SIL}
\newfontfamily\sanskritfont[Script=Devanagari]{Sanskrit 2003}
\newfontfamily\thaifont[Script=Thai]{Norasi}
\newfontfamily\thaanafont[Script=Thaana,WordSpace=2]{FreeSerif}
\newfontfamily\ethiopicfont[Script=Ethiopic]{Abyssinica SIL}
\newfontfamily\nkofont[Renderer=Graphite]{Conakry}
\parskip 1.33\baselineskip
%\newcommand\showhyphmin{\fbox{\the\lefthyphenmin\ \the\righthyphenmin}}
\begin{document}
\hyphenation{Bru-xel-les}
\noindent
\textbf{Le français}\footnote{ From \url{http://fr.wikipedia.org/wiki/Français}} est une langue romane parlée en France, dont elle est originaire (la «langue d'oïl»), ainsi qu'en Afrique francophone, au Canada (principalement au Québec, au Nouveau-Brunswick et en Ontario), en Belgique (en Région wallonne et à Bruxelles), en Suisse, au Liban, en Haïti et dans d'autres régions du monde, soit au total dans 51 pays du monde ayant pour la plupart fait partie des anciens empires coloniaux français et belge. \\
(\today)

\begin{english}
\textbf{English}\footnote{From \url{http://en.wikipedia.org/wiki/English_language}} is a West Germanic language originating in England, and the first language for most people in Australia, Canada, the Commonwealth Caribbean, Ireland, New Zealand, the United Kingdom and the United States of America (also commonly known as the Anglosphere). It is used extensively as a second language and as an official language throughout the world, especially in Commonwealth countries and in many international organisations. \\
(\today)
\end{english}

\begin{german}
\textbf{Die deutsche Sprache}\footnote{ From \url{http://de.wikipedia.org/wiki/Deutsche_Sprache}} (auch das Deutsche) gehört zum westlichen Zweig der germanischen Sprachen und ist eine der meistgesprochenen europäischen Sprachen weltweit, und gilt so als Weltsprache.\\
(\today)
\end{german}

\begin{russian}
\textbf{Русский язык} — один из восточнославянских языков, один из крупнейших языков мира, в том числе самый распространённый из славянских языков и самый распространённый язык Европы, как географически, так и по числу носителей языка как родного (хотя значительная, и географически бо́льшая, часть русского языкового ареала находится в Азии).	\\
(\today)
\end{russian}

\begin{latin}
\textbf{Lingua Latina} est lingua Indoeuropaea. Nomen ductum est de terra in paeninsula Italica quam Latine loquentes incolebant, Vetus Latium appellata sitaque inter flumen Tiberis, Volscam terram, mare Tyrrhenicum, montes Apenninos. 
Quamquam sermone nativo fungi desinit, cumque nostris diebus perpauci Latine loqui possint, lingua mortua appellari solet, multas tamen peperit linguas quae linguae romanicae vocantur, sicut Hispanicam, Francogallicam, Italicam, Lusitanam, Dacoromanicam, Gallaicam, ne omnes afferam. \\
(\today) 
\end{latin}

\begin{greek}
\textbf{Η ελληνική γλώσσα} είναι μία από τις ινδοευρωπαϊκές γλώσσες, για την
οποία έχουμε γραπτά κείμενα από τον 15ο αιώνα π.Χ. μέχρι σήμερα. Αποτελεί το
μοναδικό μέλος ενός κλάδου της ινδοευρωπαϊκής οικογένειας γλωσσών. Ανήκει
επίσης στον βαλκανικό γλωσσικό δεσμό.\\	
(\today) 
\end{greek}


\begin{hebrew}[numerals=hebrew]
\textbf{עברית} היא שפה ממשפחת השפות השמיות, הידועה כשפתו של העם היהודי, ואשר ניב מודרני שלה משמש כשפה הרשמית והעיקרית של מדינת ישראל. \\
(\today\ = \hebrewtoday)
\end{hebrew}

\begin{syriac}[numerals=abjad]
ܠܫܢܐ ܐܪܡܝܐ ܐܘ ܐܪܡܝܬ ܗܘ ܠܫܢ̈ܐ ܥܡ ܬܫܥܝܬܐ ܕ\textrm{3000} ܫܢ̈ܝܐ܂ ܗܘܐ ܠܫܢܐ ܕܡܠܟܘ̈ܬܐ ܘܬܘܕ̈ܝܬܐ܂ ܥܡ ܠܫܢܐ ܥܒܪܝܐ܄ ܗܘܐ ܠܫܢܐ ܕܣܦܪ̈ܐ ܕܕܢܝܐܝܠ ܘܥܙܪܐ ܘܗܘ ܠܫܢܐ ܚܕܢܝܐ ܕܬܠܡܘܕ܂ ܐܪܡܝܐ ܗܘܐ ܠܫܢܐ ܕܝܫܘܥ܂ ܐܕܝܘܡ܄ ܐܪܡܝܐ ܗܘ ܠܫܢܐ ܕܟܠܕ̈ܝܐ܄ ܐܬܘܪ̈ܝܐ܄ ܡܪ̈ܘܢܝܐ܄ ܘܣܘܪ̈ܝܝܐ܀ \\
(\today)
\end{syriac}

\begin{turkish}
\textbf{Türkiye Türkçesi}, Ural-Altay Dilleri içerisinde Türk dil ailesinin Oğuz Grubu'na mensup lehçedir. Anadolu, Kıbrıs, Balkanlar ve Orta Avrupa'da geniş yayılım alanı bulmuş olup, Türkiye Cumhuriyeti, Kuzey Kıbrıs Türk Cumhuriyeti, Güney Kıbrıs Rum Kesimi, Makedonya ve Kosova'nın resmî dilidir. \\
(\today = \Hijritoday)
\end{turkish}

\begin{polish}
\textbf{Język polski (polszczyzna)} należy wraz z językiem czeskim, słowackim, pomorskim (kaszubskim), dolnołużyckim, górnołużyckim oraz wymarłym połabskim do grupy języków zachodniosłowiańskich, stanowiących część rodziny języków indoeuropejskich. Ocenia się, że język polski jest językiem ojczystym około 44 milionów ludzi na świecie (w literaturze naukowej można spotkać szacunki od 40 do 48 milionów), mieszkańców Polski oraz Polaków zamieszkałych za granicą (Polonia).\\
(\today)
\end{polish}

\begin{latvian} 
\textbf{Latviešu valoda} ir dzimtā valoda apmēram 1,5 miljoniem cilvēku, galvenokārt Latvijā, kurā tā ir vienīgā valsts valoda. Lielākās latviešu valodas pratēju kopienas ārzemēs ir Austrālijā, ASV, Zviedrijā, Lielbritānijā, Vācijā, Brazīlijā, Krievijā. Latviešu valoda pieder indoeiropiešu valodu saimes baltu valodu grupai.\\
(\today)
\end{latvian}

\begin{ukrainian}
\textbf{Українська мова} — східнослов'янська мова, входить до однієї підгрупи з білоруською та російською. Подібно до цих мов українську записують кирилицею. Історично білоруська та українська мови походять з давньоруської (давньоукраїнської) — розмовної мови Київської Русі.\\
(\today)
\end{ukrainian}

\begin{sanskrit}
{\Large ससकत} पृथिव्यां प्राचीना समृद्घा वैज्ञानिकी च भाषा मन्यते । विश्ववाङ्‌मयेषु संस्कृतं श्रेष्ठरत्नम् इति न केवलं भारते अपि तु समग्रविश्वे एतद्विषये निर्णयाधिकारिभि: जनै: स्वीकृतम् । महर्षि पाणिनिना विरचिता अष्टाध्यायी इति संस्कृतव्याकरणम्‌ अधुनापि भारते विदेशेषु च भाषाविज्ञानिनां प्रेरणास्‍थानं वर्तते . संस्कृतशब्दा: एव उत्तरं दक्षिणं च भारतं संयोजयन्ति ।
\end{sanskrit}

\begin{Arabic}[]
«اعلم أنّ فنّ التاريخ فنّ عزيز المذهب، جمّ الفوائد، شريف الغاية؛ إذ هو يوقفنا على أحوال الماضين من الأمم في أخلاقهم، و الأنبياء في سيرهم، و الملوك في دولهم و سياستهم؛ حتّى تتمّ فائدة الإقتداء في ذلك لمن يرومه في أحوال الدين و الدنيا.» (ابن خلدون، المقدّمة)\\
(\today\ = \Hijritoday[0])
\end{Arabic}

\begin{farsi}
فارسی یا پارسی، (که دری، فارسی دری، و پارسی دری نیز نامیده می‌شود) زبانی است که
در کشورهای ایران، افغانستان، تاجیکستان و ازبکستان به آن سخن می‌رانند. \\
(\Jalalitoday = \Hijritoday)
\end{farsi}

\pagebreak
\begin{urdu}
اُردو ایک ہندآریائی زبان ہے جس کا تعلّق ہند یوروپی لسانی خاندان کی ہندایرانی شاخ سے ہے۔ بارہویں صدی میں ہندوستان کی مقامی زبانوں اور فارسی، عربی، اور تُرکی زبانوں کے اختلاط سے اردو وجود میں آئی۔ اردو پاکستان کی قومی زبان ہے، اور ہندوستان کی 23 سرکاری زبانوں میں سے ایک ہے۔ جنوبی ایشیا کے باہر خلیجِ فارس کے ممالک، سعودی عرب، برطانیہ، امریکہ، کنیڈا، جرمنی، ناروے، اور آسٹریلیا میں بھی جنوبی ایشیائی مہاجرین کی بڑی تعداد اردو بولتی ہے۔ \\

(\today\ مطابق \Hijritoday[0])
\end{urdu}

\begin{thai}
เป็น\wbr แผนงานเพื่อ\wbr สนับสนุน\wbr การ\wbr ร่วมกัน\wbr สร้าง, การ\wbr ร่วมกันใช้, และ\wbr การ%
ร่วมกัน\wbr พัฒนา\wbr ทรัพยากร\wbr ทาง\wbr ภาษา\wbr ของ\wbr ภาษา\wbr ไทย, บน\wbr เครือข่าย World Wide Web. แผนงานนี้\wbr มี%
จุด\wbr ประสงค์หลั\wbr กอยู่\wbr สอง\wbr ประการคือ เพื่อแก้ปัญหา\wbr กำ\wbr แพง\wbr ทาง\wbr ภาษา, และรักษา%
ไว้เพื่อ\wbr ความค\wbr งอยู่\wbr ของ\wbr ภาษา\wbr และ\wbr วัฒนธรรม\wbr ไทย. \\
(\today)
\end{thai}

\begin{divehi}\small\sloppy
ދިވެހިބަހަކީ ދިވެހިރާއްޖޭގެ ރަސްމީ ބަހެވެ. މި ބަހުން ވާހަކަ ދައްކައި އުޅެނީ ދިވެހިރާއްޖޭގެ އަހުލުވެރިންގެ އިތުރުން ހިންދުސްތާނުގެ މަލިކު ގެ
އަހުލުވެރިންނެވެ. އެބައިމީހުން މި ބަހަށް ކިޔަނީ މަހަލް ބަހެވެ. ބަހާބެހޭ މާހިރުން ދިވެހިބަސް ހިމަނުއްވައިފައިވަނީ އިންޑޯ އާރިޔަން ބަސްތަކުގެ
ތެރޭގަ އެވެ. 
\end{divehi}

%\fontspec[Script=Georgian]{DejaVu Serif}
%ქართული ენა არის საქართველოს სახელმწიფო ენა (აფხაზეთის ავტონომიურ რესპ\-უბლიკაში მის პარალელურად სახელმწიფო ენად აღიარებულია აგრეთვე აფხაზური ენა). ქართულ ენაზე 7 მილიონზე მეტი ადამიანი ლაპარაკობს.
%

\begin{amharic}
\textbf{አማርኛ} የኢትዮጵያ መደበኛ ቋንቋ ነው። ከሴማዊ ቋንቋዎች እንደ ዕብራይስጥ ወይም ዓረብኛ አንዱ ነው። እንዲያውም 27 ሚሊዮን ያህል ተናጋሪዎች እያሉት፣ አማርኛ ከአረብኛ ቀጥሎ ትልቁ ሴማዊ ቋንቋ ነው። የሚጻፈውም በግዕዝ ፊደል ነው። አማርኛ ክዓረብኛና ከዕብራይስጥ ያለው መሰረታዊ ልዩነት አንደላቲን ከግራ ወደ ቀኝ መጻፉ ነው። \\
(\today)
\end{amharic}

\begin{nko}
ߒߞߏ ߦߋ߫ ߛߓߍߟߌߞߊ߲ߞߋ ߟߋ߬ ߘߌ߫ ߝߘߊ߬ߝߌ߲߬ߠߊ߫ ߕߟߋ߬ߓߋ ߘߐ߫ ߡߊ߲߬ߘߋ߲߬ ߡߌߙߌ߲ߘߌ ߞߊ߲ ߞߊߡߊ߬߸ ߊ߬ ߣߴߊ߬ ߡߟߋߞߎߦߊߞߊ߲ ߕߐ߮ ߟߋ߬. ߞߊ߬ߕߎ߲߯ ߸ ߊ߬ ߞߘߐ ߟߋ߬ ߡߊ߲߬ߘߋ߲߫ ߝߘߏ߬ߓߊ߬ߞߊ߲ ߓߏߟߏ߲ ߓߍ߯ ߘߐ߫ ߞߏ߫: ߒ ߞߊ߲߫ ߠߋ߬ ߞߏ߫. ߝߣߊ߫߸ ߊ߬ ߦߋ߫ ߟߊߓߊ߯ߙߟߊ߫ ߟߊ߫ ߖߡߊ߬ߣߊ ߢߌ߲߬ ߠߎ߫ ߟߋ߬ ߘߐ߫ ߓߊߞߍ߭: ߖߌ߬ߣߍ߫، ߜߋ߲ߞߐ߰ߖߌ߬ߘߊ، ߊ߬ ߣߌ߫ ߡߊߟߌ߫.
\\
(\today)
\end{nko}

\end{document}
