%%
%% This is file `accsupp-test1.tex',
%% generated with the docstrip utility.
%%
%% The original source files were:
%%
%% accsupp.dtx  (with options: `test1')
%% 
%% This is a generated file.
%% 
%% Project: accsupp
%% Version: 2018/03/28 v0.5
%% 
%% Copyright (C) 2007, 2010 by
%%    Heiko Oberdiek <heiko.oberdiek at googlemail.com>
%% 
%% This work may be distributed and/or modified under the
%% conditions of the LaTeX Project Public License, either
%% version 1.3c of this license or (at your option) any later
%% version. This version of this license is in
%%    http://www.latex-project.org/lppl/lppl-1-3c.txt
%% and the latest version of this license is in
%%    http://www.latex-project.org/lppl.txt
%% and version 1.3 or later is part of all distributions of
%% LaTeX version 2005/12/01 or later.
%% 
%% This work has the LPPL maintenance status "maintained".
%% 
%% This Current Maintainer of this work is Heiko Oberdiek.
%% 
%% This work consists of the main source file accsupp.dtx
%% and the derived files
%%    accsupp.sty, accsupp.pdf, accsupp.ins, accsupp.drv,
%%    accsupp-pdftex.def, accsupp-luatex.def, accsupp-dvipdfm.def,
%%    accsupp-dvips.def,
%%    accsupp-example1.tex, accsupp-example2.tex,
%%    accsupp-test1.tex.
%% 
\NeedsTeXFormat{LaTeX2e}
\documentclass{minimal}
\makeatletter
\def\RestoreCatcodes{}
\count@=0 %
\loop
  \edef\RestoreCatcodes{%
    \RestoreCatcodes
    \catcode\the\count@=\the\catcode\count@\relax
  }%
\ifnum\count@<255 %
  \advance\count@\@ne
\repeat

\def\RangeCatcodeInvalid#1#2{%
  \count@=#1\relax
  \loop
    \catcode\count@=15 %
  \ifnum\count@<#2\relax
    \advance\count@\@ne
  \repeat
}
\def\Test{%
  \RangeCatcodeInvalid{0}{47}%
  \RangeCatcodeInvalid{58}{64}%
  \RangeCatcodeInvalid{91}{96}%
  \RangeCatcodeInvalid{123}{127}%
  \catcode`\@=12 %
  \catcode`\\=0 %
  \catcode`\{=1 %
  \catcode`\}=2 %
  \catcode`\#=6 %
  \catcode`\[=12 %
  \catcode`\]=12 %
  \catcode`\%=14 %
  \catcode`\ =10 %
  \catcode13=5 %
  \RequirePackage{accsupp}[2018/03/28]\relax
  \RestoreCatcodes
}
\Test
\csname @@end\endcsname
\end
\endinput
%%
%% End of file `accsupp-test1.tex'.
